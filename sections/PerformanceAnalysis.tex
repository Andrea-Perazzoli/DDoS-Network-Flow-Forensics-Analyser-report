\begin{figure*}[h]
	\begin{subfigure}{0.48\textwidth}
		\includegraphics[width=\textwidth]{imgs/DDoSMixed-data_analysis}
		\caption{DDoS Data Analysis} 
		\label{fig:ddos_data}
	\end{subfigure}
	\hspace*{\fill} % separation between the subfigures
	\begin{subfigure}{0.48\textwidth}
		\includegraphics[width=\textwidth]{imgs/DDoSMixed-volume_analysis}
		\caption{DDoS Volume Analysis} 
		\label{fig:ddos_volume}
	\end{subfigure}
	\caption{DDoS attack Analysis}
	\label{fig:ddos_analysis}
\end{figure*}

\section{Performance Analysis}
\label{sec:perfanalysis}
%In our tool we have developer a \textit{script} which records timestamps and duration time of big data analysis in a \texttt{.csv} file, called \texttt{PerformanceHistory.csv}. The \textit{script} mentioned before is used in order to figure out the time of analysis. 
The environment in which we tested our tool is a cluster of the University of Verona, running Ubuntu 18.04.1 LTS, it has an Intel Xeon Processor Skylake (4 cores @ 2,6 GHz) and a RAM memory of 7880 MiB for a single node, in its total configuration it has ten VCore and 60GB of RAM memory. 
We have never observed a total usage of more of 10\% CPU using \texttt{top} command and, in general, memory usage is often below 3000 MiB.
%% TODO DONE performance scritte bene guardando pagina del cluster 
%Using our script called \texttt{PerformanceAnalysis.py} we managed to plot the time elapsed to generate and analyze the datasets listed into Tab. \ref{tab:dataset_info}: results are reported into Fig. \ref{fig:datasets_statistics}.
%We can immediately see that the generation time is much greater for attack datasets: this is justifiable because of the greater size, considering multi-user disk accesses.
%Talking about analysis statistics, we can see a spike in \textit{ddos\_atk} and \textit{no\_atk} datasets. Here, long cluster timeouts during Pig Latin script execution play a great role, and we cannot have much control over it. Without that problem, analysis of the datasets would have taken much less time.
%
%\begin{figure}[ht]
%	\centering
%	\includegraphics[scale=0.49]{imgs/analysis_stat.png}
%	\caption{Performance Analysis} 
%	\label{fig:analysis_stats}
%\end{figure}
In \textbf{Tab. \ref{tab:dataset_info}} we show a summary of the dimension of the datasets we have used in our analysis. Both the \textit{LegitimateTraffic} and \textit{DoSMixed} datasets were completed in approximately 1 minute and 30 seconds. \textit{DDosMixed} was completed in 2 minutes and 27 seconds, but it's not a noticeable increment, considering the fact that it's way bigger. This justifies the parallel approach we have used with \textbf{Pig Latin}.

\begin{table}[!htbp]
\centering
\begin{tabular}{|l|l|c|}
\hline
\textbf{Dataset} & \textbf{Description}                              & \textbf{Size} \\ \hline
LegitimateTraffic         & Traffic log in normal conditions. & 316 KB        \\ 
DoSMixed        & Traffic log during a DoS attack.                   & 89 MB       \\ 
DDoSMixed        & Traffic log during a DDoS attack. & 612 MB        \\ \hline
\end{tabular}
\caption{Datasets Info}
\label{tab:dataset_info}
\end{table}
