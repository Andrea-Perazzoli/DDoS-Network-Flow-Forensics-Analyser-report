\section{Conclusions}
\label{sec:colc}
We were inspired by \cite{detection_by_path_analaysis} and \cite{ddos_forensics} for the creation of our tool, we tried to find and follow a different approach for the analysis of DoS and DDoS attacks, proposed in the cited papers, and we decide to implement and use as mathematical models quantile range outliers, standard deviation and mean. Thanks to this approach we were able to bind big data power and python in order to manipulate and show analysis results.  

Focusing on the results over our generated datasets we conclude that the precision of our tool is high in \textit{UDP flood} attacks detection, but in order to evaluate the reliability of our tool we created a testbed to recreate real conditions and record exchanged traffic. The result are quiet good, our tool recognise attacker securely involved in the DoS or DDoS genesis but as side effect it may recognise some false positive user, these may be recognised easily using standard deviation and mean or by watching its reported data in order to evaluate manually if these candidates may be or not be in the attack. High precision is also achieved by not take into consideration IP address spoofing, which is a technique that give the possibility to an attacker to create Internet Protocol (IP) packets with a false source IP address, for the purpose of impersonating another computing system. In this work, we only consider the non-address spoofing flooding attack as assumed in \cite{ddos_forensics}. 

Tests over different kind of datasets will help the reliability evaluation of our tool by analyse real attack \textit{log} files, with the attackers labeled or highlighted, but these kind of datasets are difficult to retrive. 
Possible improvements may be done by try to implement new mathematical model which uses data prediction like \cite{detection_by_path_analaysis} and implement \textit{Pig-latin} script, similar to our, in order to detect other attacks like \textit{SYN ack flood} attack which is very common like \textit{UDP flood}. 